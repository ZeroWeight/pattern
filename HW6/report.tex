\documentclass[UTF8,a4paper]{ctexart}
\usepackage[utf8]{inputenc}
\usepackage{amsmath}
\usepackage{pdfpages}
\usepackage{graphicx}
\usepackage{wrapfig}
\usepackage{listings}
\usepackage{xcolor}
\lstset{
    numbers=left, 
    numberstyle= \tiny, 
    keywordstyle= \color{ blue!70},
    commentstyle= \color{red!50!green!50!blue!50}, 
    frame=shadowbox, % 阴影效果
    rulesepcolor= \color{ red!20!green!20!blue!20} ,
    escapeinside=``, % 英文分号中可写入中文
    xleftmargin=2em,xrightmargin=2em, aboveskip=1em,
    framexleftmargin=2em
} 
\title{模式识别作业6}
\author{张蔚桐\ 2015011493\ 自55}
\begin {document}
\maketitle
\section{}
\subsection{}
首先由分类面形状可以知道(C)图必然为线性核函数得到的结果,同时也可以得到(D)是精细径向基($\sigma=0.1$)得带的结果。同时,观察支持向量的位置和分类面和线性分类面的差异可以看出,(F)是三次多项式核函数的结果,而(E)是粗糙径向基得到的结果($\sigma=1$),()是中等径向基得到的结果($\sigma=0.5$),()是二次多项式核得到的结果。总结如下

\begin{table}
\caption{不同分类面对应的核函数}
\centering
\begin{tabular}{|c|c|c|c|}
\hline
a &  & b & \\
\hline
c &线性核函数& d &$\sigma=0.1$径向基\\
\hline
e &  & f & \\
\hline
\end{tabular}
\end{table}
\subsection{}
观察数据点性质可以看出这些数据点还是可以线性可分的,因此选择线性核函数比较合理,因为比较简单,同时分类效果足够好,又不会产生过拟合情况。
\section{}
\end{document}