\documentclass[UTF8,a4paper]{ctexart}
\usepackage[utf8]{inputenc}
\usepackage{amsmath}
\usepackage{pdfpages}
\usepackage{graphicx}
\usepackage{wrapfig}
\usepackage{listings}
\usepackage{xcolor}
\lstset{
    numbers=left, 
    numberstyle= \tiny, 
    keywordstyle= \color{ blue!70},
    commentstyle= \color{red!50!green!50!blue!50}, 
    frame=shadowbox, % 阴影效果
    rulesepcolor= \color{ red!20!green!20!blue!20} ,
    escapeinside=``, % 英文分号中可写入中文
    xleftmargin=2em,xrightmargin=2em, aboveskip=1em,
    framexleftmargin=2em
} 
\title{模式识别作业2}
\author{张蔚桐\ 2015011493\ 自55}
\begin {document}
\maketitle
\section{}
\subsection{}
根据题目中的提示,我们将误差函数分别对$\mathbf{w}$和$w_0$求导可以得到
\begin{equation}\displaystyle{
\frac{\partial E}{\partial \mathbf{w}}=\sum_{i=1}^n((\mathbf{w}^T\mathbf{x}_i+w_0-t_i)\mathbf{x}_i^T) }
\end{equation}\begin{equation}
\displaystyle{
\frac{\partial E}{\partial w_0}=\sum_{i=1}^n(\mathbf{w}^T\mathbf{x}_i+w_0-t_i)}
\end{equation}
整理可得误差函数极小值点的方程
\begin{equation}
\displaystyle{
\frac{\partial E}{\partial \mathbf{w}}=
\mathbf{w}^T\sum_{i=1}^n(\mathbf{x}_i\mathbf{x}_i^T)+w_0\sum_{i=1}^n(\mathbf{x}_i^T)-\sum_{i=1}^n(t_i\mathbf{x}_i^T)\ =0}
\label{w} 
\end{equation}\begin{equation}
\displaystyle{
\frac{\partial E}{\partial w_0}=\mathbf{w}^T\sum_{i=1}^n(\mathbf{x}_i)+nw_0-\sum_{i=1}^n(t_i)=0}
\label{w2}
\end{equation}
根据题目中的设定$t_1=\frac{n}{n_1}$以及$t_2=-\frac{n}{n_2}$可以迅速得到$\sum_{i=1}^n(t_i)=0$

因此化简\ref{w2}式得到
\begin{equation}
\displaystyle{
w_0=-\frac{\mathbf{w}^T\sum_{i=1}^n(\mathbf{x}_i)}{n}=-\mathbf{w}^T\mathbf{m}}
\label{c1}
\end{equation}
由\ref{c1},此问得证
\subsection{}
将\ref{c1}带入\ref{w}可以得到如下表述
\begin{equation}\begin{aligned}
\displaystyle{
\mathbf{w}^T\sum_{i=1}^n(\mathbf{x}_i\mathbf{x}_i^T)-\mathbf{w}^T\mathbf{m}\sum_{i=1}^n(\mathbf{x}_i^T)-\sum_{i=1}^n(t_i\mathbf{x}_i^T)=0}\\
\displaystyle{
\mathbf{w}^T\sum_{i=1}^n((\mathbf{x}_i-\mathbf{m})\mathbf{x}_i^T)=\sum_{i=1}^n(t_i\mathbf{x}_i^T)}
\label{w3}
\end{aligned}\end{equation}
下面着重研究\ref{w3}式右侧的表述,按照类别可以划分为
\begin{equation}\begin{aligned}
\displaystyle{
\sum_{i=1}^n(t_i\mathbf{x}_i^T)=\frac{n}{n_1}\sum_{\mathcal{C}_1}(\mathbf{x}_i^T)-\frac{n}{n_2}\sum_{\mathcal{C}_2}(\mathbf{x}_i^T)}\\
=n(\mathbf{m_1-m_2})
\label{s}
\end{aligned}\end{equation}
结合\ref{s},对\ref{w3}式取转置可得
\begin{equation}
\displaystyle{
\sum_{i=1}^n(\mathbf{x}_i(\mathbf{x}_i-\mathbf{m})^T)\mathbf{w}=n(\mathbf{m_1-m_2})}
\label{w4}
\end{equation}
下面处理\ref{w4}式$\mathbf{w}$前系数问题,根据统计学知识,我们可以得到$\mathbf{m}=\frac{n_1\mathbf{m_1}+n_2\mathbf{m_2}}{n}$,由此,将\ref{w4}式左侧系数按照类别展开,得到
\begin{equation}\begin{aligned}
\sum_{i=1}^n(\mathbf{x}_i(\mathbf{x}_i-\mathbf{m})^T)=
\sum_{\mathcal{C}_1}(\mathbf{x}_i(\mathbf{x}_i-\frac{n_1\mathbf{m_1}+n_2\mathbf{m_2}}{n})^T)\\
+\sum_{\mathcal{C}_2}(\mathbf{x}_i(\mathbf{x}_i-\frac{n_1\mathbf{m_1}+n_2\mathbf{m_2}}{n})^T)\\
=	\frac{n_1}{n}\sum_{\mathcal{C}_1}(\mathbf{x}_i(\mathbf{x}_i-\mathbf{m_1})^T)
  	+\frac{n_2}{n}\sum_{\mathcal{C}_1}(\mathbf{x}_i(\mathbf{x}_i-\mathbf{m_2})^T)\\
	+\frac{n_1}{n}\sum_{\mathcal{C}_2}(\mathbf{x}_i(\mathbf{x}_i-\mathbf{m_1})^T)
  	+\frac{n_2}{n}\sum_{\mathcal{C}_2}(\mathbf{x}_i(\mathbf{x}_i-\mathbf{m_2})^T)\\
=	\frac{n_1}{n}\sum_{\mathcal{C}_1}((\mathbf{x}_i-\mathbf{m_1})(\mathbf{x}_i-\mathbf{m_1})^T)
	+\frac{n_1}{n}\sum_{\mathcal{C}_1}(\mathbf{m_1}(\mathbf{x}_i-\mathbf{m_1})^T)\\
	+\frac{n_2}{n}\sum_{\mathcal{C}_1}((\mathbf{x}_i-\mathbf{m_1})(\mathbf{x}_i-\mathbf{m_2})^T)
	+\frac{n_2}{n}\sum_{\mathcal{C}_1}(\mathbf{m_1}(\mathbf{x}_i-\mathbf{m_2})^T)\\
	+\frac{n_1}{n}\sum_{\mathcal{C}_2}((\mathbf{x}_i-\mathbf{m_2})(\mathbf{x}_i-\mathbf{m_1})^T)
	+\frac{n_1}{n}\sum_{\mathcal{C}_2}(\mathbf{m_2}(\mathbf{x}_i-\mathbf{m_1})^T)\\
	+\frac{n_2}{n}\sum_{\mathcal{C}_2}((\mathbf{x}_i-\mathbf{m_2})(\mathbf{x}_i-\mathbf{m_2})^T)
	+\frac{n_2}{n}\sum_{\mathcal{C}_2}(\mathbf{m_2}(\mathbf{x}_i-\mathbf{m_2})^T)\\
=	\frac{n_1}{n}\sum_{\mathcal{C}_1}((\mathbf{x}_i-\mathbf{m_1})(\mathbf{x}_i-\mathbf{m_1})^T)
	+\frac{n_2}{n}\sum_{\mathcal{C}_2}((\mathbf{x}_i-\mathbf{m_2})(\mathbf{x}_i-\mathbf{m_2})^T)\\
	+\frac{n_2n_1}{n}\mathbf{m_1}(\mathbf{m_1}-\mathbf{m_2})^T
	-\frac{n_1n_2}{n}\mathbf{m_2}(\mathbf{m_1}-\mathbf{m_2})^T\\
	+\frac{n_2}{n}\sum_{\mathcal{C}_1}((\mathbf{x}_i-\mathbf{m_1})(\mathbf{x}_i-\mathbf{m_1})^T)
	+\frac{n_2}{n}\sum_{\mathcal{C}_1}((\mathbf{x}_i-\mathbf{m_1})(\mathbf{m_1}-\mathbf{m_2})^T)\\
	+\frac{n_1}{n}\sum_{\mathcal{C}_2}((\mathbf{x}_i-\mathbf{m_2})(\mathbf{x}_i-\mathbf{m_2})^T)
	+\frac{n_1}{n}\sum_{\mathcal{C}_2}((\mathbf{x}_i-\mathbf{m_2})(\mathbf{m_2})^T-\mathbf{m_1})^T)\\
=	S_w+\frac{n_1n_2}{n}S_B
\label{le}
\end{aligned}\end{equation}
于是可以得证本题
\begin{equation}
(S_w+\frac{n_1n_2}{n}S_B)\mathbf{w}=n(\mathbf{m_1-m_2})
\label{q2}
\end{equation}
\subsection{}
设若$\mathbf{w}=kS_w^{-1}(\mathbf{m_2-m_1})$满足上式\ref{q2},将其带入\ref{q2}式可得
\begin{equation}
k\frac{n_1n_2}{n}S_BS_w^{-1}(\mathbf{m_2-m_1})=(n+k)(\mathbf{m_1-m_2})
\end{equation}
上式成立因此可得
\begin{equation}
S_BS_w^{-1}(\mathbf{m_2-m_1})=(n+k+k\frac{n_1n_2}{n})(\mathbf{m_2-m_1})
\end{equation}
经过上式分析可以发现,本题相当于证明$S_BS_w^{-1}$矩阵定有$(\mathbf{m_2-m_1})$的特征向量且其特征值不为$n$
\begin{equation}\begin{aligned}
S_BS_w^{-1}(\mathbf{m_2-m_1})=(\mathbf{m_2-m_1})(\mathbf{m_2-m_1})^TS_w^{-1}(\mathbf{m_2-m_1})\\
=(\mathbf{m_2-m_1})((\mathbf{m_2-m_1})^TS_w^{-1}(\mathbf{m_2-m_1}))
\end{aligned}
\end{equation}
可以发现右侧括号中为一二次型,结果为标量。因此可以证明$S_BS_w^{-1}$矩阵定有$(\mathbf{m_2-m_1})$的特征向量。并且显然一般情况下其不为$n$
因此此问得证。
\section{}
其中MATLAB代码已经在附件中包含,平台采用的是MATLAB R2015b,执行时,请采用以下顺序
\lstset{language=Matlab}
\begin{lstlisting}
% load data, please do first
dataload; 

%%%%%Fisher method%%%%%%
% Fisher method
mainFisher; 

% Check for Fisher method
% Please run after Fisher method
checkFisher; 

%%%%%%Logistic method%%%%%%
% Logistic method
mainLogistic; 

% Check for Logistic method
% Please run after Logistic method
checkLogistic; 
%You can decide the order of Logistic method 
% and Fisher method freely%

\end{lstlisting}
使用Logistics方法测试集准确性在0.97以上(三次测试),使用Fisher方法测试集准确性在0.96以上
\end{document}