\documentclass[UTF8,a4paper]{ctexart}
\usepackage[utf8]{inputenc}
\usepackage{amsmath}
\usepackage{pdfpages}
\usepackage{graphicx}
\usepackage{wrapfig}
\usepackage{listings}
\title{模式识别作业2}
\author{张蔚桐\ 2015011493\ 自55}
\begin {document}
\maketitle
\section{}
\subsection{}
根据题目中的提示,我们将误差函数分别对$\mathbf{w}$和$w_0$求导可以得到
\begin{equation}\displaystyle{
\frac{\partial E}{\partial \mathbf{w}}=\sum_{i=1}^n((\mathbf{w}^T\mathbf{x}_i+w_0-t_i)\mathbf{x}_i^T) }
\end{equation}\begin{equation}
\displaystyle{
\frac{\partial E}{\partial w_0}=\sum_{i=1}^n(\mathbf{w}^T\mathbf{x}_i+w_0-t_i)}
\end{equation}
整理可得误差函数极小值点的方程
\begin{equation}
\displaystyle{
\frac{\partial E}{\partial \mathbf{w}}=
\mathbf{w}^T\sum_{i=1}^n(\mathbf{x}_i\mathbf{x}_i^T)+w_0\sum_{i=1}^n(\mathbf{x}_i^T)-\sum_{i=1}^n(t_i\mathbf{x}_i^T)\ =0}
\label{w} 
\end{equation}\begin{equation}
\displaystyle{
\frac{\partial E}{\partial w_0}=\mathbf{w}^T\sum_{i=1}^n(\mathbf{x}_i)+nw_0-\sum_{i=1}^n(t_i)=0}
\label{w2}
\end{equation}
根据题目中的设定$t_1=\frac{n}{n_1}$以及$t_2=-\frac{n}{n_2}$可以迅速得到$\sum_{i=1}^n(t_i)=0$

因此化简\ref{w2}式得到
\begin{equation}
\displaystyle{
w_0=-\frac{\mathbf{w}^T\sum_{i=1}^n(\mathbf{x}_i)}{n}=-\mathbf{w}^T\mathbf{m}}
\label{c1}
\end{equation}
由\ref{c1},此问得证
\subsection{}
将\ref{c1}带入\ref{w}可以得到如下表述
\begin{equation}\begin{aligned}
\displaystyle{
\mathbf{w}^T\sum_{i=1}^n(\mathbf{x}_i\mathbf{x}_i^T)-\mathbf{w}^T\mathbf{m}\sum_{i=1}^n(\mathbf{x}_i^T)-\sum_{i=1}^n(t_i\mathbf{x}_i^T)=0}\\
\displaystyle{
\mathbf{w}^T\sum_{i=1}^n((\mathbf{x}_i-\mathbf{m})\mathbf{x}_i^T)=\sum_{i=1}^n(t_i\mathbf{x}_i^T)}
\label{w3}
\end{aligned}\end{equation}
下面着重研究\ref{w3}式右侧的表述,按照类别可以划分为
\begin{equation}\begin{aligned}
\displaystyle{
\sum_{i=1}^n(t_i\mathbf{x}_i^T)=\frac{n}{n_1}\sum_{\mathcal{C}_1}(\mathbf{x}_i^T)-\frac{n}{n_2}\sum_{\mathcal{C}_2}(\mathbf{x}_i^T)}\\
=n(\mathbf{m_1-m_2})
\label{s}
\end{aligned}\end{equation}
结合\ref{s},对\ref{w3}式取转置可得
\begin{equation}
\displaystyle{
\sum_{i=1}^n(\mathbf{x}_i(\mathbf{x}_i-\mathbf{m})^T)\mathbf{w}=n(\mathbf{m_1-m_2})}
\label{w4}
\end{equation}
下面处理\ref{w4}式$\mathbf{w}$前系数问题,根据统计学知识,我们可以得到$\mathbf{m}=\frac{n_1\mathbf{m_1}+n_2\mathbf{m_2}}{n}$,由此,将\ref{w4}式左侧系数按照类别展开,得到

\end{document}