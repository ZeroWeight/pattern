\documentclass[UTF8,a4paper]{ctexart}
\usepackage[utf8]{inputenc}
\usepackage{amsmath}
\usepackage{pdfpages}
\usepackage{graphicx}
\usepackage{wrapfig}
\usepackage{listings}
\usepackage{xcolor}
\lstset{
    numbers=left, 
    numberstyle= \tiny, 
    keywordstyle= \color{ blue!70},
    commentstyle= \color{red!50!green!50!blue!50}, 
    frame=shadowbox, % 阴影效果
    rulesepcolor= \color{ red!20!green!20!blue!20} ,
    escapeinside=``, % 英文分号中可写入中文
    xleftmargin=2em,xrightmargin=2em, aboveskip=1em,
    framexleftmargin=2em
} 
\title{模式识别作业4}
\author{张蔚桐\ 2015011493\ 自55}
\begin {document}
\maketitle
\section{神经网络的训练}
如下图所示是隐节点分别为5,10,20,40,100的情况的测试集的confusionTest matrix

从各个图中可以看出,随着隐节点的增多,混淆矩阵指示的准确性提高,训练误差逐渐降低,分析错误率同时降低,收敛速度也得到了提高。但是,当隐节点增多到一定程度后,出现了过拟合的情况,上面的各个评估指数的变化逐渐不明显
\begin{figure}
\includegraphics[width=\textwidth]{5confusionTest.jpg}
\caption{5隐节点测试集混淆矩阵}
\includegraphics[width=\textwidth]{5performance.jpg}
\caption{5隐节点训练误差曲线}
\end{figure}
\begin{figure}
\includegraphics[width=\textwidth]{10confusionTest.jpg}
\caption{10隐节点测试集混淆矩阵}
\includegraphics[width=\textwidth]{10performance.jpg}
\caption{10隐节点训练误差曲线}
\end{figure}
\begin{figure}
\includegraphics[width=\textwidth]{20confusionTest.jpg}
\caption{20隐节点测试集混淆矩阵}
\includegraphics[width=\textwidth]{20performance.jpg}
\caption{20隐节点训练误差曲线}
\end{figure}
\begin{figure}
\includegraphics[width=\textwidth]{40confusionTest.jpg}
\caption{40隐节点测试集混淆矩阵}
\includegraphics[width=\textwidth]{40performance.jpg}
\caption{40隐节点训练误差曲线}
\end{figure}
\begin{figure}
\includegraphics[width=\textwidth]{100confusionTest.jpg}
\caption{100隐节点测试集混淆矩阵}
\includegraphics[width=\textwidth]{100performance.jpg}
\caption{100隐节点训练误差曲线}
\end{figure}
\clearpage
\section{神经网络准确性和线性方法的比较}

\end{document}