\documentclass[UTF8,a4paper]{ctexart}
\usepackage[utf8]{inputenc}
\usepackage{amsmath}
\usepackage{pdfpages}
\usepackage{graphicx}
\usepackage{wrapfig}
\usepackage{listings}
\usepackage{xcolor}
\lstset{
    numbers=left, 
    numberstyle= \tiny, 
    keywordstyle= \color{ blue!70},
    commentstyle= \color{red!50!green!50!blue!50}, 
    frame=shadowbox, % 阴影效果
    rulesepcolor= \color{ red!20!green!20!blue!20} ,
    escapeinside=``, % 英文分号中可写入中文
    xleftmargin=2em,xrightmargin=2em, aboveskip=1em,
    framexleftmargin=2em
} 
\title{模式识别作业3}
\author{张蔚桐\ 2015011493\ 自55}
\begin {document}
\maketitle
\section{}
考察采用$\lambda(\hat{\theta},\theta)=(\hat{\theta}-\theta)^2$的期望风险可得
\begin{equation}
R(\hat{\theta}|\mathbf{X})=\int_\theta(\hat{\theta}-\theta)^2P(\theta|\mathbf{X})\rm{d}\theta
\label{R}
\end{equation}
 对\ref{R}式求导可得
\begin{equation}
\frac{\rm{d}R(\hat{\theta}|\mathbf{X})}{\rm{d}\hat{\theta}}=\int_\theta2(\hat{\theta}-\theta)P(\theta|\mathbf{X})\rm{d}\theta
\label{dR}
\end{equation}
期望风险最小化可以得到\ref{dR}式为0,因此可以得到
\begin{equation}
\int_\theta\theta P(\theta|\mathbf{X})\rm{d}\theta=\hat{\theta} \int_\theta{P(\theta|\mathbf{X})\rm{d}\theta}
\label{eqn}
\end{equation}
\ref{eqn}等式右侧系数为归一化系数,因此可直接得到结论
\begin{equation}
\hat{\theta}=\int_\theta\theta P(\theta|\mathbf{X})\rm{d}\theta=E(\theta|\mathbf{X})
\label{eqn}
\end{equation}
\section{}

\end{document}